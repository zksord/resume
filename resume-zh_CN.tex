% !TEX TS-program = xelatex
% !TEX encoding = UTF-8 Unicode
% !Mode:: "TeX:UTF-8"

\documentclass{resume}
\usepackage{zh_CN-Adobefonts_external} % Simplified Chinese Support using external fonts (./fonts/zh_CN-Adobe/)
%\usepackage{zh_CN-Adobefonts_internal} % Simplified Chinese Support using system fonts
\usepackage{linespacing_fix} % disable extra space before next section
\usepackage{hyperref}
\usepackage{cite}

\begin{document}
\pagenumbering{gobble} % suppress displaying page number

\name{麦广灿}

% {E-mail}{mobilephone}{homepage}
% be careful of _ in emaill address
\contactInfo{csgcmai@life.hkbu.edu.hk}{+001 (517) 402-9186}{\url{www.comp.hkbu.edu.hk/~csgcmai}}
% {E-mail}{mobilephone}
% keep the last empty braces!
%\contactInfo{xxx@yuanbin.me}{(+86) 131-221-87xxx}{}
 
\section{\faGraduationCap\  教育背景}
\datedsubsection{\textbf{香港浸会大学}, 九龙塘, 香港}{2013年9月 -- 至今}
\textit{博士}\ 计算机科学, 预计 2017 年 11 月毕业\\
\datedsubsection{\textbf{密歇根州立大学}, 东兰辛, 美国}{2016年2月 -- 2016年8月 (预期)}
\textit{访问学者}\ \href{http://biometrics.cse.msu.edu/}{模式识别及图像处理实验室},计算机科学与工程系\\
\datedsubsection{\textbf{华南理工大学}, 广州, 中国}{2010年9月 -- 2013年7月}
\textit{工程学士}\ 计算机科学与技术(全英联合班,三年毕业)\\
- GPA 3.50/4.00, 85/100 排名 5/28\\
\textit{*}\ 班级共33人, 选拔自全校申请者,其中5名被淘汰\\

\section{\faUsers\ 主要实习/项目经历}
\datedsubsection{\textbf{密歇根州立大学}, 东兰辛, 美国}{2016年2月 -- 2016年8月 (预期)}
\role{访问学者}{\textit{指导老师}: \href{http://cse.msu.edu/~jain}{Prof. Anil K. Jain}}
- 主要研究及开发基于深度学习的人脸图像重构算法
\datedsubsection{\textbf{香港浸会大学}\ 九龙塘,香港}{2013年9月 -- 至今}
\role{博士生}{指导老师: \href{http://www.comp.hkbu.edu.hk/~pcyuen}{Prof. Pong C. Yuen}}
- 主要研究生物识别系统(如人脸、虹膜、指纹)安全及多模态生物识别系统, 开发了一个能优化多模态生物识别系统安全性和精度的二进制特征融合算法(具体见学术论文)
\datedsubsection{\textbf{华南理工大学}\  广州, 中国}{2012年5月 -- 2013年5月}
\role{科研本科生}{指导老师: 张星明教授}
- 主要开发基于Android、移动摄像头及PC服务器的移动视频监控系统,个人负责视频流的提取、传输及Android客户端开发\\
- 软件著作权一项,美新杯2012广东赛区一等奖\\
\datedsubsection{\textbf{华南理工大学}\  广州,中国}{2011年5月 -- 2012年6月}
\role{科研本科生}{指导老师: 唐韶华教授,賴晓铮博士}
- 主要开发基于Xilinx FPGA的低功耗密码芯片及控制系统, 个人负责控制系统及外设传感器等模块的开发(包括GPRS串口模块、红外传感器等)\\
- 第三届开源硬件大赛三等奖, 广东省首届大学生创新创业大赛“最受欢迎项目”


\section{\faLink\ 社团/社会服务}
\datedsubsection{\textbf{学生代表}\  香港浸会大学理学院}{2014年 -- 2016年}
- 组织理学院及计算机系各类学生活动,如香港十大杰出青年科学家分享会、Lunchbite等
\datedsubsection{\textbf{主席}\  内地学生学者联谊会,香港浸会大学}{2014年 -- 2016年}
- 组织香港浸会大学内地学生各类活动,如“学在香港”说明会、“香港浸会大学、香港中文大学、香港城市大学三校游艇联谊会”等。与校内外学生组织,企业,各青年社团建立了长期友好的合作关系


\section{\faHeartO\ 主要获奖情况}
\begin{itemize}
	\item \datedline{\textit{四年博士生全额奖学金},香港浸会大学}{2013 - 2017年}
	\item \datedline{\textit{海外学术访问奖学金}, 香港浸会大学计算机科学系,郭一苇郭钟宝芬伉俪研究院发展基金}{2016 年}
	\item \datedline{\textit{三等奖(第四名)}, ACM大学生编程竞赛 (香港站)}{2014年6月}
	\item \datedline{\textit{国家励志奖学金}}{2011年,2012年}
	\item \datedline{\textit{三好学生},华南理工大学}{2011年,2012年}
	\item \datedline{\textit{提名奖}, 华南理工大学十大卓越团队(共20支队伍被提名)}{2012年}
	\item \datedline{\textit{最受欢迎项目奖},首届广东省大学生创新创业年会}{2012年10月}
	\item \datedline{\textit{一等奖},美新杯物联网创新创业大赛广东赛区}{2012年9月}
	\item \datedline{\textit{三等奖},第三届开源硬件与嵌入式大赛 (20/218)}{2012年6月}
	\item \datedline{\textit{二等奖},广东省机器人大赛}{2011年9月}
	\item \datedline{\textit{优秀学生干部},华南理工大学}{2011年}
\end{itemize}

\section{\faMicrophone \ 学术演讲}
\begin{itemize}
	\item\datedline{\textbf{Guangcan Mai}, \href{http://www.comp.hkbu.edu.hk/~csgcmai/papers/BinaryTemplateFusion_BTAS2015.pptx}{Fusing Binary Templates for Multi-biometric Cryptosystems}}{2015年9月}
	- 华盛顿特区,美国
\end{itemize}

\section{\faBook\ 学术论文}
\begin{itemize}
	\item \textbf{Guangcan Mai}, Meng-Hui Lim, Pong C Yuen, Fusing Binary Templates for Multi-biometric Cryptosystems. \textit{BTAS2015}
	\item \textbf{Guangcan Mai}, Meng-Hui Lim, Pong C Yuen, Binary Feature Fusion for Discriminative and Secure Multi-biometric Cryptosystems, 获邀投稿至 \textit{Image and Vision Computing} (在审)
	\item Meng-Hui Lim, Sunny Verma, \textbf{Guangcan Mai}, Pong C Yuen, Learning Discriminability-Preserving Histogram Representation from Unordered Features for Multibiometric Feature-Fused-Template Protection, 投稿至 \textit{Pattern Recognition} (在审)
\end{itemize}


\section{\faCogs\ IT 技能}
% increase linespacing [parsep=0.5ex]
\begin{itemize}[parsep=0.5ex]
	\item 编程语言: MATLAB, C/C++, Python, Java, PHP, VerlogHDL (根据熟悉程度排序)
	\item 平台及工具:Windwos, Linux, Android,  Vim, \LaTeX, Office, Git, Visual Studio, MXNET, FPGA
\end{itemize}

\section{\faInfo\ 其他}
% increase linespacing [parsep=0.5ex]
\begin{itemize}[parsep=0.5ex]
  \item GitHub: \url{https://github.com/csgcmai}
  \item 语言: 普通话,粤语,英语
\end{itemize}

%% Reference
%\newpage
%\bibliographystyle{IEEETran}
%\bibliography{mycite}
\end{document}
