% !TEX TS-program = xelatex
% !TEX encoding = UTF-8 Unicode
% !Mode:: "TeX:UTF-8"

\documentclass{resume}
\usepackage{zh_CN-Adobefonts_external} % Simplified Chinese Support using external fonts (./fonts/zh_CN-Adobe/)
%\usepackage{zh_CN-Adobefonts_internal} % Simplified Chinese Support using system fonts
\usepackage{linespacing_fix} % disable extra space before next section
\usepackage{hyperref}
\usepackage{cite}

\begin{document}
\pagenumbering{gobble} % suppress displaying page number

\name{麦广灿}

% {E-mail}{mobilephone}{homepage}
% be careful of _ in emaill address
\contactInfo{csgcmai@life.hkbu.edu.hk}{+001 (517)402-9186}{\url{www.comp.hkbu.edu.hk/~csgcmai}}
% {E-mail}{mobilephone}
% keep the last empty braces!
%\contactInfo{xxx@yuanbin.me}{(+86) 131-221-87xxx}{}
 
\section{\faGraduationCap\  教育背景}
\datedsubsection{\textbf{密歇根州立大学}, 密歇根, 美国}{2016.2 -- 2016.8}
\textit{访学学者}\ \href{http://biometrics.cse.msu.edu}{模式识别及图像处理实验室}\\
\textit{指导老师}\ \href{http://cse.msu.edu/~jain}{Prof. Anil K. Jain}
\datedsubsection{\textbf{香港浸会大学}, 九龙塘, 香港}{2013.9 -- 至今}
\textit{博士}\ 计算机科学, 预计 2017 年 11 月毕业\\
\textit{研究方向}\ 生物识别, 生物识别安全, 多模态生物识别\\
\textit{指导老师}\ \href{http://www.comp.hkbu.edu.hk/~pcyuen}{Prof. Pong C Yuen}
\datedsubsection{\textbf{华南理工大学}, 广州, 中国}{2010.9 -- 2013.7}
\textit{学士}\ 计算机科学与技术(全英联合班,三年毕业)\\
\textit{*}\ 班级共33人, 选拔自全校申请者\\

\section{\faBook\ 学术论文}
\begin{itemize}
	\item \textbf{Guangcan Mai}, Meng-Hui Lim, Pong C Yuen, Fusing Binary Templates for Multi-biometric Cryptosystems. \textit{BTAS2015} (oral)
	\item \textbf{Guangcan Mai}, Meng-Hui Lim, Pong C Yuen, Binary Feature Fusion for Discriminative and Secure Multi-biometric Cryptosystems, invited to submit to \textit{Image and Vision Computing} (under revision)
	\item Meng-Hui Lim, Sunny Verma, \textbf{Guangcan Mai}, Pong C Yuen, Learning Discriminability-Preserving Histogram Representation from Unordered Features for Multibiometric Feature-Fused-Template Protection, submitted to \textit{Pattern Recognition}
\end{itemize}

%\section{\faUsers\ 实习/项目经历}
%\datedsubsection{\textbf{黑科技公司} 上海}{2015年3月 -- 2015年5月}
%\role{实习}{经理: 高富帅}
%xxx后端开发
%\begin{itemize}
%  \item 实现了 xxx 特性
%  \item 后台资源占用率减少8\%
%  \item xxx
%\end{itemize}
%
%\datedsubsection{\textbf{分布式科学上网姿势}}{2014年6月 -- 至今}
%\role{Golang, Linux}{个人项目,和富帅糕合作开发}
%\begin{onehalfspacing}
%分布式负载均衡科学上网姿势, https://github.com/cyfdecyf/cow
%\begin{itemize}
%  \item 修复了连接未正常关闭导致文件描述符耗尽的 bug
%  \item 使用Chord 哈希 URL, 实现稳定可靠地分流
%  \item xxx (尽量使用量化的客观结果)
%\end{itemize}
%\end{onehalfspacing}
%
%\datedsubsection{\textbf{\LaTeX\ 简历模板}}{2015 年5月 -- 至今}
%\role{\LaTeX, Python}{个人项目}
%\begin{onehalfspacing}
%优雅的 \LaTeX\ 简历模板, https://github.com/billryan/resume
%\begin{itemize}
%  \item 容易定制和扩展
%  \item 完善的 Unicode 字体支持,使用 \XeLaTeX\ 编译
%  \item 支持 FontAwesome 4.5.0
%\end{itemize}
%\end{onehalfspacing}

% Reference Test
%\datedsubsection{\textbf{Paper Title\cite{zaharia2012resilient}}}{May. 2015}
%An xxx optimized for xxx\cite{verma2015large}
%\begin{itemize}
%  \item main contribution
%\end{itemize}

\section{\faCogs\ IT 技能}
% increase linespacing [parsep=0.5ex]
\begin{itemize}[parsep=0.5ex]
  \item 编程语言: MATLAB, C/C++, Python, Java, PHP, VerlogHDL (根据熟悉程度排序)
  \item 平台及工具:Windwos, Linux, Vim, \LaTeX, Git
\end{itemize}

\section{\faHeartO\ 主要获奖情况}
\begin{itemize}
	\item \datedline{\textit{国家励志奖学金}}{2011年,2012年}
	\item \datedline{\textit{三等奖(第四名)}, ACM大学生编程竞赛 (香港站)}{2014年6月}
	\item \datedline{\textit{一等奖},美新杯物联网创新创业大赛广东赛区}{2012年9月}
	\item \datedline{\textit{三等奖},第三届开源硬件与嵌入式大赛 (20/218)}{2012年6月}
	\item \datedline{\textit{二等奖},广东省机器人大赛}{2011年9月}
\end{itemize}

\section{\faInfo\ 其他}
% increase linespacing [parsep=0.5ex]
\begin{itemize}[parsep=0.5ex]
  \item GitHub: \url{https://github.com/csgcmai}
  \item 语言: 普通话,粤语,英语
\end{itemize}

%% Reference
%\newpage
%\bibliographystyle{IEEETran}
%\bibliography{mycite}
\end{document}
